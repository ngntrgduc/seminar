\begin{frame}{Kết quả thực nghiệm: Mạng NeuMiss}
% Tập dữ liệu được sinh ra từ phân phối chuẩn đa biến, với $Y$ được tính bằng hàm tuyến tính của $X$, cùng với $50\%$ dữ liệu bị khuyết ngẫu nhiên (MCAR) ở mỗi đặc trưng trong tập dữ liệu và nhiễu $\varepsilon$ tuân theo phân phối chuẩn với tỷ lệ signal-to-noise (SNR) bằng~$10$.
% (tỷ lệ nhiễu nhỏ hơn $10$ lần so với ``tín hiệu'' $Y$).

% Ta sử dụng metric $R$ bình phương ($R^2$), để tính tỷ lệ của độ biến thiên cho biến phụ thuộc $Y$ được giải thích bởi biến độc lập $X$, từ đó đánh giá độ hiệu quả của mô hình. 
    \begin{table}[h!]
    \centering
    \setlength{\tabcolsep}{10pt}
    \begin{tabular}{lcccccc}
    \toprule
    & \multicolumn{2}{c}{\textbf{Train Set}} & \multicolumn{2}{c}{\textbf{Validation Set}} & \multicolumn{2}{c}{\textbf{Test Set}} \\
    \cmidrule(lr){2-3}\cmidrule(lr){4-5}\cmidrule(lr){6-7}
     & $R^2$ & MSE & $R^2$ & MSE & $R^2$ & MSE \\
    \midrule
    NeuMiss depth-10 & 0.8072 & 0.208 & 0.7482 & 0.2808 & 0.7578 & 0.2674 \\
    \bottomrule
    \end{tabular}
    % \captionsetup{justification=centering, width=\linewidth}
    \caption{Hiệu suất của NeuMiss với độ sâu 10.}
    \label{tab:performance}
    \end{table}
Với Bayes rate cho tập train là 0.8205 và tập test là 0.8078.
\end{frame}

\begin{frame}{Kết quả thực nghiệm: Có và khi không có residual connection}
    \begin{table}[h!]
    \centering
    \setlength{\tabcolsep}{7.3pt}
    \begin{tabular}{p{5.3cm}cccccc}
    \toprule
    & \multicolumn{2}{c}{\textbf{Train Set}} & \multicolumn{2}{c}{\textbf{Validation Set}} & \multicolumn{2}{c}{\textbf{Test Set}} \\
    \cmidrule(lr){2-3}\cmidrule(lr){4-5}\cmidrule(lr){6-7}
     & $R^2$ & MSE & $R^2$ & MSE & $R^2$ & MSE \\
    \midrule
    \raggedright Có residual connection & 0.8116 & 0.2033 & 0.7424 & 0.2873 & 0.7511 & 0.2747 \\
    \raggedright Không có residual connection & 0.8257 & 0.1880 & 0.7708 & 0.2556 & 0.7641 & 0.2604 \\
    \bottomrule
    \end{tabular}
    % \captionsetup{justification=centering, width=\linewidth}
    \caption{Hiệu suất của NeuMiss với độ sâu 10 khi có và không có residual connection.}
    \label{tab:performance_residual}
    \end{table}
\end{frame}

\begin{frame}{Kết quả thực nghiệm: Với các độ sâu khác nhau}
\begin{table}[h!]
\centering
\setlength{\tabcolsep}{8pt}
\begin{tabular}{ccccccc}
\toprule
\textbf{Độ sâu} & \multicolumn{2}{c}{\textbf{Train Set}} & \multicolumn{2}{c}{\textbf{Validation Set}} & \multicolumn{2}{c}{\textbf{Test Set}} \\
\cmidrule(lr){2-3}\cmidrule(lr){4-5}\cmidrule(lr){6-7}
 & $R^2$ & MSE & $R^2$ & MSE & $R^2$ & MSE \\
\midrule
1  & 0.7823 & 0.2349 & 0.7500 & 0.2788 & 0.7678 & 0.2563 \\
5  & 0.7959 & 0.2202 & 0.7492 & 0.2796 & 0.7586 & 0.2664 \\
10 & 0.8191 & 0.1951 & 0.7548 & 0.2735 & 0.7676 & 0.2565 \\
15 & 0.8164 & 0.1980 & 0.7624 & 0.2650 & 0.7636 & 0.2610 \\
20 & 0.8157 & 0.1988 & 0.7286 & 0.3027 & 0.7431 & 0.2836 \\
\bottomrule
\end{tabular}
% \captionsetup{justification=centering, width=\linewidth}
\caption{Hiệu suất của NeuMiss với các độ sâu khác nhau.}
\label{tab:performance-depths}
\end{table}
\end{frame}

\begin{frame}{Kết quả thực nghiệm: Với các tỷ lệ dữ liệu khuyết khác nhau}
\begin{table}[h!]
\centering
\setlength{\tabcolsep}{8pt}
\begin{tabular}{ccccccc}
\toprule
\textbf{Tỷ lệ khuyết} & \multicolumn{2}{c}{\textbf{Train Set}} & \multicolumn{2}{c}{\textbf{Validation Set}} & \multicolumn{2}{c}{\textbf{Test Set}} \\
\cmidrule(lr){2-3}\cmidrule(lr){4-5}\cmidrule(lr){6-7}
  & $R^2$ & MSE & $R^2$ & MSE & $R^2$ & MSE \\
\midrule
0.1\% & 0.8133 & 0.2014 & 0.7565 & 0.2716 & 0.7506 & 0.2752 \\
0.2\% & 0.8145 & 0.2001 & 0.7533 & 0.2751 & 0.7517 & 0.2741 \\
0.5\% & 0.8191 & 0.1952 & 0.7700 & 0.2565 & 0.7682 & 0.2558 \\
0.8\% & 0.7973 & 0.2187 & 0.7505 & 0.2782 & 0.7572 & 0.2680 \\
\bottomrule
\end{tabular}
% \captionsetup{justification=centering, width=\linewidth}
\caption{Hiệu suất của NeuMiss độ sâu 10 với các tỷ lệ khuyết khác nhau.}
\label{tab:performance_missing_rates}
\end{table}

% Ta nhận thấy rằng dù tỷ lệ khuyết nhiều nhưng NeuMiss vẫn cho ra kết quả tốt.
\end{frame}


\begin{frame}{Kết quả thực nghiệm: Với các tuỳ chỉnh số lượng mẫu và đặc trưng khác nhau}
\begin{table}[h!]
\centering
\setlength{\tabcolsep}{6pt}
\begin{tabular}{cccccccc}
\toprule
\textbf{Mẫu} & \textbf{Đặc trưng} & \multicolumn{2}{c}{\textbf{Train Set}} & \multicolumn{2}{c}{\textbf{Validation Set}} & \multicolumn{2}{c}{\textbf{Test Set}} \\
\cmidrule(lr){3-4}\cmidrule(lr){5-6}\cmidrule(lr){7-8}
 & & $R^2$ & MSE & $R^2$ & MSE & $R^2$ & MSE \\
\midrule
\multirow{2}{*}{100} & 10 & 0.8609 & 0.1581 & -1.6307 & 2.0416 & -12.7133 & 7.1884 \\
                     & 20 & 0.9450 & 0.0596 & -324.8323 & 205.5099 & -418.3801 & 224.8063 \\
\midrule
\multirow{2}{*}{1000} & 10 & 0.5976 & 0.4539 & -1.5842 & 2.8056 & -0.4908 & 1.6403 \\
                      & 20 & 0.7279 & 0.2953 & -42.1458 & 41.3359 & -64.2313 & 43.3147 \\
\midrule
\multirow{2}{*}{5000} & 10 & 0.8186 & 0.2037 & 0.7657 & 0.2481 & 0.7715 & 0.2491 \\
                      & 20 & 0.8929 & 0.1189 & -0.2029 & 1.3849 & -0.1616 & 1.3378 \\
\midrule
\multirow{2}{*}{10000} & 10 & 0.8138 & 0.2008 & 0.7489 & 0.2800 & 0.7568 & 0.2684 \\
                       & 20 & 0.8783 & 0.1336 & 0.1887 & 0.9035 & 0.2765 & 0.8806 \\
\midrule
\multirow{2}{*}{50000} & 10 & 0.8249 & 0.1933 & 0.8124 & 0.2071 & 0.8001 & 0.2232 \\
                       & 20 & 0.8056 & 0.2153 & 0.7145 & 0.3151 & 0.7282 & 0.3021 \\
\bottomrule
\end{tabular}
% \caption{Hiệu suất của NeuMiss độ sâu 10 với các tuỳ chỉnh số lượng mẫu và đặc trưng khác nhau cho tập dữ liệu.}
\label{tab:performance_grouped_samples_features}
\end{table}
\end{frame}

\begin{frame}{Kết quả thực nghiệm: Với các phương pháp khác}
    
\begin{table}[h!]
\centering
\setlength{\tabcolsep}{10pt}
\begin{tabular}{lcc}
\toprule
& \multicolumn{1}{c}{\textbf{Train Set}} 
& \multicolumn{1}{c}{\textbf{Test Set}} \\
% \cmidrule(lr){2-3}\cmidrule(lr){4-5}\cmidrule(lr){6-7}
 % & $R^2$ & MSE & $R^2$ & MSE \\
\midrule
NeuMiss depth-10 & \textbf{0.8072}  & \textbf{0.7578} \\
KNN imputer + LR & 0.6929 & 0.7018 \\
Simple imputer + LR & 0.6574 & 0.6478 \\
SoftImpute + LR & 0.7604 & 0.7524 \\
MissForest + LR & 0.7708 & 0.7442 \\
MICE + LR & 0.7496 & 0.7313 \\
Simple imputer + MLP regressor & 0.8022 & 0.7330 \\

\bottomrule
\end{tabular}
\caption{Hiệu suất của NeuMiss độ sâu 10 với các phương pháp khác.}
\label{tab:performance-others}
\end{table}

% Qua các kết quả này, ta có thể thấy mạng NeuMiss tỏ ra vượt trội hơn so với các phương pháp điền khuyết truyền thống.
\end{frame}
